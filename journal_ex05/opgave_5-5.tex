{
    \newcommand{\labelprefix}{src5-5}

    \subsection{Count Ones}

    % Intro 
    \subsubsection{Introduktion}

    I denne del af øvelsen vil vi forsøge at tælle antallet af 1'ere i en array af data vha. et \texttt{FOR-LOOP}.

    % Design
    \subsubsection{Design og implementering}

    Vores implementation baseret på opgaven ses i \coderef[\labelprefix-ent]{ex05/vhdl/count_ones.vhd} og \coderefsimple[\labelprefix-arch]{ex05/vhdl/count_ones.vhd}.
    Processen har \texttt{SW} i sin sensitivity liste, hvilket vil sige den kører hver gang \texttt{SW} ændrer sig.
    \texttt{FOR-LOOP}'et gennemgår alle bits i input signalet og tæller en intern process variabel op hvis der er et 1-tal.
    Talte 1'ere assignes til et signal i arkitekturen som bruges til at vise tallet på et 7-segment display.

    \includecode[\labelprefix-ent]{ex05/vhdl/count_ones.vhd}{Interface til count\_ones entity}{linerange={6-11}}

    \includecode[\labelprefix-arch]{ex05/vhdl/count_ones.vhd}{Arkitektur til optælling af 1'ere vha. \texttt{FOR-LOOP}}{linerange={13-37}}

    % Results
    \subsubsection{Resultater}

    % TODO RTL view?
    % TODO funktionel simulation af count_ones ELLER billeder af at den virker?
    Vi ser at count\_ones virker som forventet og tæller hvor mange 1'ere der er på input signalet (altså hvor mange switches er slået til).

    % Discussion
    \subsubsection{Diskussion}

    \texttt{FOR-LOOP} giver god mening når man vil gentage ikke-triviel kode et bestemt antal gange.
    Som set tidligere gør sådanne loops det også nemt at lave generiske komponenter.
    I denne opgave kunne man nemt introducere en \texttt{GENERIC} til at skalere input signalet og loopet.
    % TODO diskussion om rtl'et?

    % Conclusion
    \subsubsection{Konklusion}

    Loops er meget nyttige. :-) % TODO ingen ide om hvad man kan konkludere
}