{
\newcommand{\labelprefix}{src6-1}

\subsection{Counter - One Digit}

% Intro 
\subsubsection{Introduktion}

Denne opgave handler om at skabe en tæller der går fra \texttt{[0:9]}. For at gøre dette vil vi udnytte nogle af \texttt{IEEE.numeric\_std.ALL} biblioteket, der blandet andet indholder \emph{edge detection}.

% Design
\subsubsection{Design og implementering}

Selve tælleren blev skrevet vha. \texttt{process} og \texttt{case} metoder.



\includecode[\labelprefix]{ex06/vhdl/count_onedigit.vhd}{[Count one digit entity] Her ses Count one digit entitien, hvor input og output definers}{linerange=5-16}

\includecode[\labelprefix]{ex06/vhdl/count_onedigit.vhd}{[Count one digit arkitektur] Her ses Count one digit arkitekturen, hvor bla. \texttt{mode} og \texttt{reset} implementers}{linerange=18-66}

\includecode[\labelprefix]{ex06/vhdl/multi_counter_tester.vhd}{[Test bench: Multi Counter]Test setup der viser et test setup for vores tæller.}{linerange=5-41}

\includecode[\labelprefix]{ex06/vhdl/test_bench.vhd}{[Test bench: Multi Counter Tester]Test setuppet \coderef[\labelprefix]{ex06/vhdl/test_bench.vhd}, sat op i vores test bench.}{linerange=37-45}


% Results
\subsubsection{Resultater}


% Discussion
\subsubsection{Diskussion}

% Conclusion
\subsubsection{Konklusion}

}