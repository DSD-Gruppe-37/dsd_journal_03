{
    \newcommand{\labelprefix}{src5-2}

    \subsection{Guess Game}

    % Intro 
    \subsubsection{Introduktion}

    Målet med denne opgave var at skabe et gættespil, hvor en spiller forsøger at gætte en talværdi bestemt af en anden person.
    Spillet skal alt imens kunne give et visuelt feedback, dette baseret på spillernes input.
    Dette lægger grund for brugen af et register der går systemet sekventielt, i stedet for kombinatorisk som de andre vi tidligere har arbejdet med.

    % Design
    \subsubsection{Design og implementering}

    For at spillet kan muligøres, skal der laves en række elementer som udgøre selve spillets kerne.
    Alle entities indeholder de almindeligt brugte \texttt{library} og \texttt{use} statements som er udeladt for at være lidt mere kortfattet.
    De entities der skal bruges er som følger:

    \subsubsection*{Guess Game}
    Delblokkene blev samlet i en \texttt{guess\_game} entity der blev opbygget strukturelt og udnyttede muligheden for at lave interne signaler.

    Da koden blev meget lang, er den delt op i flere stykker.
    \includecode[\labelprefix]{ex05/vhdl/guess_game.vhd}{[Guess Game Entity]Interface til guess\_game entity.}{linerange={0-14}}

    \includecode[\labelprefix]{ex05/vhdl/guess_game.vhd}{[Guess Game arkitektur og interne signaler]Her ses begyndelsen på \texttt{Guess Game} arkitekturen, samt de signaler der oprettes}{linerange={15-21}}

    \includecode[\labelprefix]{ex05/vhdl/guess_game.vhd}{[Guess Game arkitekturen del 1]Her ses de interne forbindelser mellem de oprettede elementmer.}{linerange={23-62}}

    \includecode[\labelprefix]{ex05/vhdl/guess_game.vhd}{[Guess Game arkitekturen del 2]Slutningen af koden, hvor de sidste elementer forbindes, og evt. ubrugte porte bliver der taget hånd om.}{linerange={63-87}}

    \subsubsection*{Latch}
    Et register der kan fastholde en værdi. Vores implementation ses på \coderef[\labelprefix]{ex05/vhdl/latch.vhd} er generisk og kan derfor gemme værdier af forskellig størrelse. Denne har til formål at gemme det korrekte tal, når der trykkes på \texttt{set}.

    \includecode[\labelprefix]{ex05/vhdl/latch.vhd}{Latch entity og arkitektur}{linerange={5-24}}

    \subsubsection*{Compare}
    Compare logikken der er vist i \coderef[\labelprefix]{ex05/vhdl/CompareLogic.vhd} er sat op vha. en \texttt{IF} og \texttt{WHEN-ELSE} statement, der sammenligninger de inputs der er tilstede---men kun når \texttt{en} er aktiv, hvilket \texttt{process()} søgrer for.

    \includecode[\labelprefix]{ex05/vhdl/CompareLogic.vhd}{Compare entity og arkitektur}{linerange={6-33}}

    \subsubsection*{Multiplexers}
    \includecode[\labelprefix]{ex05/vhdl/mux.vhd}{Fælles mux entity}{linerange={7-14}}
    Det var tydeligt at binary to 7-segement decoderen bestod af multiplexers - hvilket gør at følgende multiplexers kan skrives på samme måde.
    \includecode[\labelprefix]{ex05/vhdl/mux.vhd}{Mux2 arkitektur}{linerange={37-48}}

    Derfor kunne den let skrives på samme måde, hvilket ses i \coderef[\labelprefix]{ex05/vhdl/mux.vhd}

    \includecode[\labelprefix]{ex05/vhdl/mux.vhd}{Mux4 arkitektur}{linerange={19-32}}
    Ved at udvide multiplexeren, kunne der ligeledes laves en multiplexer der håndterede visning af 10'ere og 1'ere på de 2 hexdisplays.

    % Results
    \subsubsection{Resultater}

    Guess Game blev testet i en testbench, som ses på \coderef[\labelprefix]{ex05/vhdl/guess_game_tester.vhd}.

    \includecode[\labelprefix]{ex05/vhdl/guess_game_tester.vhd}{Guess Game testbench}{}

    I både \figref{ex5-2-full-rtl} og \figref{ex5-2-full-postmapping}, ses implementationen af de forskellige elementer.

    \dsdfig{ex5-2-full-rtl}{12cm}{RTL view: Guess game}
    \dsdfig{ex5-2-full-postmapping}{0.9\linewidth}{Postmapping view: Guess game}

    I fig. \ref{tab:ex5-2-rtls} ses udsnit af de forskellige elementers RTL views.

    \begin{figure}[H]
        \centering
        \dsdsubfig{ex5-2-Latch-rtl}{Latch entity}{}
        \hfill
        \dsdsubfig{ex5-2-compare-rtl}{Compare entity}{}
        \vskip\baselineskip
        \dsdsubfig{ex5-2-muxfour-rtl}{Muxfour entity}{}
        \hfill
        \dsdsubfig{ex5-2-muxtwo-rtl}{Muxtwo entity}{}
        \caption[RTL view: Guess Game elementer]{Udsnit af RTL views fra de forskellige elementer i Guess Game}\label{tab:ex5-2-rtls}
    \end{figure}

    \subsubsection*{Latches}

    Vi ser ud fra compilation report i Listing \coderefsimple[\labelprefix]{ex05/latchreport1p.txt} at der bliver lavet 8 inferred latches.
    \includecode[\labelprefix]{ex05/latchreport1p.txt}{Udsnit af output report}{}

    Desuden har vi testet spillet på vores DE2 board og observeret input-output relationerne som set i fig. \ref{tab:ex5-2-response}.

    \begin{figure}[H]
        \centering
        \dsdsubfig{ex5-2_SecretValue}{Secret value / correct input}{0.2}
        \dsdsubfig{ex5-2_Correct}{Input correct}{0.2}
        \dsdsubfig{ex5-2_loInput}{Low input} {0.2}
        \dsdsubfig{ex5-2_loGuess}{Input too low} {0.2}
        \dsdsubfig{ex5-2_hiInput}{High input}{0.2}
        \dsdsubfig{ex5-2_hiGuess}{Input too high}{0.2}
        \caption[HEX view: Guess Game]{De forskellige HEX outputs vist.}\label{tab:ex5-2-response}
    \end{figure}

    % Discussion
    \subsubsection{Diskussion}

    Som vi så i compilation report i Listing \coderefsimple[\labelprefix]{ex05/latchreport1p.txt} blev der lavet 8 inferred latches, hvilket stemmer overens med vores forventning. Hele ideen med vores latch komponent er jo at den skal gemme en værdi som et register på \emph{n} bits, hvor vi i dette tilfælde har sat \(n = 8\) og derfor forventede 8 latches.

    % Conclusion
    \subsubsection{Konklusion}

    Vi har haft success med at bruge latches til at gemme værdi, hvilket er en meget grundlæggende ting at kunne for at lave mere komplekse sekventielle systemer.
}
