{
    \newcommand{\labelprefix}{src5-1}

\subsection{Binary to 7-Segment Decoder Using “CASE”}

% Intro 
\subsubsection{Introduktion}

I denne opgave vil vi igen skrive en Binary to 7-segment decoder, dog vil vi bruge \texttt{case} statements istedet.

Funktionaliteten bør i teorien være det sammen som hvis vi brugte \texttt{when} statements.

% Design
\subsubsection{Design og implementering}
Decoderen blev designet på samme måde som i øvelse 4, og med de samme betingelsesværdier, dog med \texttt{case} i stedet. De to syntaxer minder om hinanden, men der er dog nogle forskelle. Resulatet  af koden ses i \coderef[\labelprefix]{ex05/vhdl/bin2sevenseg.vhd}

\includecode[\labelprefix]{ex05/vhdl/bin2sevenseg.vhd}{Binary to 7-segment entity og arkitektur}{linerange={4-36}}

% Results
\subsubsection{Resultater}

\dsdfig{ex5-1-rtl}{0.4\linewidth}{RTL view: Binary to 7-segment}
\dsdfig{ex5-1-post-mapping}{0.8\linewidth}{Postmapping view (udsnit): Binary to 7-segment}





For at teste Funktionaliteten blev blev der også lavet en testcase, denne ses i \tabref{7segmentTestCase}, og dens testbench kode ses i \coderef[\labelprefix]{ex05/vhdl/test_bench.vhd}

\dsdtab{
    \begin{tabular}{p{4cm}p{4cm}p{4cm}}\toprule
        Input & Forventet display & Reelt display \\\midrule
        \texttt{0101} & \texttt{5} & \texttt{5} \\\midrule
        \texttt{0111} & \texttt{7} & \texttt{7} \\\midrule
        \texttt{1111} & \texttt{F} & \texttt{F} \\\bottomrule
    \end{tabular}
    }{Test case for binary to 7-segment }{7segmentTestCase}
    
    \includecode[\labelprefix]{ex05/vhdl/test_bench.vhd}{[Test bench: Binary to 7-segment] Test bench setup, hvori Binary to 7-segment afprøves}{linerange={57-64}}

% Discussion
\subsubsection{Diskussion}

Det ses på \figref{ex5-1-rtl} at den igen består af en række multiplexers. Der er altså ikke den store forskel på \texttt{with} og \texttt{case} i dette tilfælde. Post-mapping viewet i \figref{ex5-1-post-mapping} viser heller ikke nogen forskel, i forhold til sidste øvelse.

% Conclusion
\subsubsection{Konklusion}

Der er ingen reel forskel på, om man implementere en binary to 7-segment vha. \texttt{case} eller \texttt{with} statements - begge er lige velfungerende. Det bør dog undersøges om der er en forskel i ren praksis.
        
}