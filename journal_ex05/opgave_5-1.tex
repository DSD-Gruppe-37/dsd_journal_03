{
    \newcommand{\labelprefix}{src5-1}

    \subsection{Binary to 7-Segment Decoder Using “CASE”}

    % Intro 
    \subsubsection{Introduktion}

    I denne opgave vil vi igen skrive en Binary to 7-segment decoder. Vi har tidligere lavet en med \texttt{WITH-SELECT} syntaks, denne gang laves den i en \texttt{process} med en \texttt{CASE} statement.

    Funktionaliteten bør være den sammen som vores forrige kombinatoriske implementation med \texttt{WITH-SELECT} statements.

    % Design
    \subsubsection{Design og implementering}

    Decoderen blev designet på samme måde som i Øvelse 4, og med de samme betingelsesværdier, dog med en \texttt{CASE} i stedet. De to syntakser minder om hinanden, men der er dog nogle forskelle. Implementationen ses i \coderef[\labelprefix]{ex05/vhdl/test_bench.vhd}

    \includecode[\labelprefix]{ex05/vhdl/bin2sevenseg.vhd}{Binary to 7-segment entity og arkitektur}{linerange={5-36}}

    % Results
    \subsubsection{Resultater}

    \dsdfig{ex5-1-rtl}{0.5\linewidth}{RTL view: Binary to 7-segment}
    \dsdfig{ex5-1-post-mapping}{0.8\linewidth}{Postmapping view (udsnit): Binary to 7-segment}

    For at teste funktionaliteten blev der også lavet en testcase, denne ses i \tabref{7segmentTestCase}, og dens testbench kode ses i \coderef[\labelprefix]{ex05/vhdl/test_bench.vhd}

    \dsdtab{
        \begin{tabular}{p{4cm}p{4cm}p{4cm}}\toprule
            Input & Forventet display & Reelt display \\\midrule
            \texttt{0101} & \texttt{5} & \texttt{5} \\\midrule
            \texttt{0111} & \texttt{7} & \texttt{7} \\\midrule
            \texttt{1111} & \texttt{F} & \texttt{F} \\\bottomrule
        \end{tabular}
        }{Test case for binary to 7-segment }{7segmentTestCase}
        
        \includecode[\labelprefix]{ex05/vhdl/test_bench.vhd}{[Test bench: Binary to 7-segment] Test bench setup, hvori Binary to 7-segment afprøves}{linerange={59-66}}

    % Discussion
    \subsubsection{Diskussion}

    Det ses på \figref{ex5-1-rtl} at den ligesom i det kombinatoriske design består af en række multiplexers.
    Der er altså ikke den store forskel på \texttt{WITH-SELECT} og \texttt{CASE} i dette tilfælde. Post-mapping viewet i \figref{ex5-1-post-mapping} viser heller ikke nogen forskel, i forhold til sidste øvelse.

    % Conclusion
    \subsubsection{Konklusion}

    Der er ingen reel forskel på, om man implementere en binary to 7-segment vha. \texttt{CASE} eller \texttt{WITH-SELECT} statements---begge er lige velfungerende. Det bør dog undersøges om der er en forskel i ren praksis.
            
}