{
\newcommand{\labelprefix}{{src5-4}}    
\subsection{8-input NAND Gate}

% Intro 
\subsubsection{Introduktion}

I denne opgave designede vi en \textit{n}-input \texttt{NAND}-gate. 
Formålet med øvelsen er at udnytte VHDL's loopfunktion, og desuden afprøve \texttt{generic} funktionerne. Sidst nævnte muliggøre \textit{n}-inputtet, da vi med denne selv kan vælge antallet af inputs. 


% Design
\subsubsection{Design og implementering}

For at lave en 8-input \texttt{NAND}, satte vi \texttt{generic} størrelsen til \texttt{+8}.

\includecode[\labelprefix]{ex05/vhdl/nand8.vhd}{Generic input NAND entity og arkitektur}{}

% Results
\subsubsection{Resultater}
\dsdfig{ex5-4-32_bit_NAND}{0.4\linewidth}{RTL view: 32 bit NAND}

% Discussion
\subsubsection{Diskussion}
% Conclusion
\subsubsection{Konklusion}}