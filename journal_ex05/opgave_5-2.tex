{
    \newcommand{\labelprefix}{src5-2}

    \subsection{Guess Game}

    % Intro 
    \subsubsection{Introduktion}

    Målet med denne opgave var at skabe et gættespil, hvor en spiller forsøger at gætte en talværdi bestemt af en anden person.
    Spillet skal alt imens kunne give et visuelt feedback, dette baseret på spillernes input.
    Dette lægger grund for brugen af et register der går systemet sekventielt, i stedet for kombinatorisk som de andre vi tidligere har arbejdet med.

    % Design
    \subsubsection{Design og implementering}

    For at spillet kan muligøres, skal der laves en række elementer som udgøre selve spillets kerne.
    Alle entities indeholder de almindeligt brugte \texttt{library} og \texttt{use} statements som er udeladt for at være lidt mere kortfattet.
    De entities der skal bruges er som følger:

    \paragraph{Latch}
    Et register der kan fastholde en værdi. Vores implementation ses på \coderef[\labelprefix]{ex05/vhdl/latch.vhd} er generisk og kan derfor gemme værdier af forskellig størrelse. Denne har til formål at gemme det korrekte tal, når der trykkes på \texttt{set}.

    \includecode[\labelprefix]{ex05/vhdl/latch.vhd}{Latch entity og arkitektur}{linerange={5-24}}

    \paragraph{Compare}
    Compare logikken der er vist i \coderef[\labelprefix]{ex05/vhdl/CompareLogic.vhd} er sat op vha. en \texttt{IF} og \texttt{WHEN-ELSE} statement, der sammenligninger de inputs der er tilstede---men kun når \texttt{en} er aktiv, hvilket \texttt{process()} søgrer for.

    \includecode[\labelprefix]{ex05/vhdl/CompareLogic.vhd}{Compare entity og arkitektur}{linerange={6-28}}

    \paragraph{Multiplexers}
    \includecode[\labelprefix]{ex05/vhdl/mux.vhd}{Fælles mux entity}{linerange={7-24}}
    Det var tydeligt at binary to 7-segement decoderen bestod af multiplexers - hvilket gør at følgende multiplexers kan skrives på samme måde.
    \includecode[\labelprefix]{ex05/vhdl/mux.vhd}{Mux2 arkitektur}{linerange={47-58}}

    % TODO reference til de to første mux figurer?
    Derfor kunne den let skrives på samme måde, hvilket ses i \coderef[\labelprefix]{ex05/vhdl/mux.vhd}

    \includecode[\labelprefix]{ex05/vhdl/mux.vhd}{Mux4 arkitektur}{linerange={29-42}}
    Ved at udvide multiplexeren, kunne der ligeledes laves en multiplexer der håndterede visning af 10'ere og 1'ere på de 2 hexdisplays.

    \paragraph{Guess Game}
    Delblokkene blev samlet i en \texttt{guess\_game} entity der blev opbygget strukturelt og udnyttede muligheden for at lave interne signaler.

    Da koden blev meget lang, er den delt op i flere stykker.
    \includecode[\labelprefix]{ex05/vhdl/guess_game.vhd}{[Guess Game Entity]Her ses selve entitydeklerationen, hvor alle input og output oprettes.}{linerange={4-13}}


    \includecode[\labelprefix]{ex05/vhdl/guess_game.vhd}{[Guess Game arkitektur og interne signaler]Her ses begyndelsen på \texttt{guess\_game} arkitekturen, samt de signaler der oprettes}{linerange={15-20}}

    \includecode[\labelprefix]{ex05/vhdl/guess_game.vhd}{[Guess Game arkitekturen del 1]Her ses de interne forbindelser mellem de oprettede elementer.}{linerange={23-62}}

    \includecode[\labelprefix]{ex05/vhdl/guess_game.vhd}{[Guess Game arkitekturen del 2]Slutningen af koden, hvor de sidste elementer forbindes, og evt. ubrugte porte bliver der taget hånd om.}{linerange={63-87}}

    % Results
    \subsubsection{Resultater}

    I både \figref{ex5-2-full-rtl} og \figref{ex5-2-full-postmapping}, ses implementationen af de forskellige elementer.

    \dsdfig{ex5-2-full-rtl}{1.0\linewidth}{RTL view: Guess game}
    \dsdfig{ex5-2-full-postmapping}{1.0\linewidth}{Postmapping view: Guess game}

    \begin{figure}[H]
        \centering
        % TODO opdateres til at matche implementering
        \dsdsubfig{ex5-2-Latch-rtl}{Latch entity}
        \hfill
        \dsdsubfig{ex5-2-compare-rtl}{Compare entity}
        \vskip\baselineskip
        \dsdsubfig{ex5-2-muxfour-rtl}{Muxfour entity}
        \hfill
        \dsdsubfig{ex5-2-muxtwo-rtl}{Muxtwo entity}
        \caption[RTL view: Guess Game elementer]{Udsnit af RTL views fra de forskellige elementer i Guess Game}\label{tab:ex5-2-rtls}
    \end{figure}

    I fig. \ref{tab:ex5-2-rtls} ses udsnit af de forskellige elementers RTL views.

    % TODO rettelse af testbench til guess_game
    \includecode[\labelprefix]{ex05/vhdl/test_bench.vhd}{[Test bench: Guess Game]Test bench setup af Guess game't}{linerange=46-55}

    % Discussion
    \subsubsection{Diskussion}

    % TODO beskrivelse af latches i design
    % TODO billede af compilation med inferred latches(?)

    % Conclusion
    \subsubsection{Konklusion}

    % TODO tilføjelse om registre og latches
    Denne opgave har vist hvor effektivt VHDL er til at sammensætte elementer. Alt der skal til er en god kodestandard, og et par signaler. Justereinger i et enkelt element kan nemt foretages og implementeres i en hovedfil.
}
