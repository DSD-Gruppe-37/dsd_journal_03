{
    \newcommand{\labelprefix}{src5-5}

    \subsection{Count Ones}

    % Intro 
    \subsubsection{Introduktion}

    I denne del af øvelsen vil vi forsøge at tælle antallet af 1'ere i en array af data vha. et \texttt{FOR-LOOP}.

    % Design
    \subsubsection{Design og implementering}

    Vores implementation baseret på opgaven ses i \coderef[\labelprefix-ent]{ex05/vhdl/count_ones.vhd} og \coderefsimple[\labelprefix-arch]{ex05/vhdl/count_ones.vhd}.
    Processen har \texttt{SW} i sin sensitivity liste, hvilket vil sige den kører hver gang \texttt{SW} ændrer sig.
    \texttt{FOR-LOOP}'et gennemgår alle bits i input signalet og tæller en intern process variabel op hvis der er et 1-tal.
    Talte 1'ere assignes til et signal i arkitekturen som bruges til at vise tallet på et 7-segment display.

    \includecode[\labelprefix-ent]{ex05/vhdl/count_ones.vhd}{Interface til count\_ones entity}{linerange={6-11}}

    \includecode[\labelprefix-arch]{ex05/vhdl/count_ones.vhd}{Arkitektur til optælling af 1'ere vha. \texttt{FOR-LOOP}}{linerange={13-37}}

    % Results
    \subsubsection{Resultater}

    På RTL-viewet for vores \texttt{count\_ones} på fig. \ref{tab:ex5-5-count-ones} ses det at komponent laves som en kæde af additioner og multiplexere, der bruger vores input som deres selects.

    \begin{figure}[H]
        \centering
        \dsdsubfig{ex5-5_rtl-view_left}{Venstre side af RTL-viewet}{}
        \dsdsubfig{ex5-5_rtl-view_right}{Højre side af RTL-viewet}{}
        \caption[RTL view af count\_ones]{Venstre og højre side af count\_ones RTL-view}\label{tab:ex5-5-count-ones}
    \end{figure}

    Vi har testet vores 1-tæller på DE2 boardet og har fået resultaterne som set i \tabref{oneCounterTest}.

    \dsdtab{
        \begin{tabular}{p{4cm}p{4cm}p{4cm}}\toprule
            Input & Forventet på display & Reelt på display \\\midrule
            \texttt{01001100} & \texttt{3} & \texttt{3} \\\midrule
            \texttt{10110011} & \texttt{5} & \texttt{5} \\\midrule
        \end{tabular}
        }{Test case for 1-tælleren}{oneCounterTest}

    På fig. \ref{tab:ex5-5-count-ones-de2} ses vores test på DE2 svarende til test casene ovenfor.

    \begin{figure}[H]
        \centering
        \dsdsubfig{ex5-5_3}{Test case 1}{}
        \dsdsubfig{ex5-5_5}{Test case 2}{}
        \caption[count\_ones test cases]{Test på DE2 board}\label{tab:ex5-5-count-ones-de2}
    \end{figure}

    % Discussion
    \subsubsection{Diskussion}

    Vi ser at count\_ones virker som forventet og tæller hvor mange 1'ere der er på input signalet (altså hvor mange switches er slået til).

    \texttt{FOR-LOOP} giver god mening når man vil gentage ikke-triviel kode et bestemt antal gange.
    Som set tidligere gør sådanne loops det også nemt at lave generiske komponenter.
    I denne opgave kunne man nemt introducere en \texttt{GENERIC} til at skalere input signalet og loopet.

    % Conclusion
    \subsubsection{Konklusion}

    Vi har set at man i loops også sagtens kan bruge behavioural statements så som \texttt{IF} og dermed kan bruge loops til at skrive kompleks kode der skal gentages en bestemt antal gange.
}