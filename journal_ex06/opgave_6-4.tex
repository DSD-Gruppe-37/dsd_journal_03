{
    \newcommand{\labelprefix}{src6-4}

\subsection{Alarm Watch}

% Intro 
\subsubsection{Introduktion}

Ved at tilføje nogle enkelte moduler kan uret fra tidligere udvides med en sammenligningsfunktionalitet --- dette vil kunne bruges til at lave tidsindstillet alarm. 

% Design
\subsubsection{Design og implementering}

Vi implementerede \texttt{watch} entity og en række \texttt{bin2sevenseg} entities, og tilføjede desuden 3 nye elementer: \texttt{Multiplexer}, \texttt{InputLimiter} og \texttt{Compare}.

\includecode[ent]{ex06/vhdl/alarm_watch_tester.vhd}{Alarm watch entity}{linerange={6-28}}
\includecode[arch1]{ex06/vhdl/alarm_watch_tester.vhd}{[Alarm watch arkitektur] Signaldekleration for alarmen}{linerange={39-49}}
\includecode[arch2]{ex06/vhdl/alarm_watch_tester.vhd}{[Alarm watch arkitektur] Her ses hvordan selve uret (\coderefsimple[watch1]{ex06/vhdl/watch.vhd}) og multiplexeren implementeres}{linerange={51-66,68-96}}
\includecode[arch2]{ex06/vhdl/alarm_watch_tester.vhd}{[Alarm watch arkitektur] Her ses input limiteren, compare module og et quaddisplay}{linerange={100-111,114-123,126-139}}
\dsdfig{ex6-4-rtl}{0.9\textwidth}{RTL view: Alarm watch}

\subsubsection*{Quad display}
\includecode[arch2]{ex06/vhdl/quadBin2Sevenseq.vhd}{[Quad HEX display] Samling af fire displays}{firstline=6}

Da de fire outputdisplays består af \emph{slice}'s af et enkelt signal, valgte vi at samle dem i en entity for sig selv - dette gjorde også at RTL view'et i \figref{ex6-4-rtl} blev nær identisk med det IBD (fig. 6 i øvelsesvejlednignen) der var udleveret. 

\subsubsection*{Compare logic}
\includecode[compare]{ex06/vhdl/compareTime.vhd}{Alarm watch arkitektur: Compare}{linerange={5-29}}
Modulet i \coderef[compare]{ex06/vhdl/compareTime.vhd} består blot af en simpel komparator og en \texttt{IF/ELSE} statement der aktivere et \texttt{alarm}output, hvis indholdet at de to inputtede \texttt{STD\_LOGIC\_VECTOR} er ens.

\subsubsection*{Input limiting}
\includecode[input]{ex06/vhdl/inputLimiting.vhd}{Alarm watch arkitektur: Input limiting}{linerange={5-57}}
Tanken bag dette modul, var at begrænse inputværdierne, så den maksimale værdi var \texttt{[23:59]}. 
Der blev lavet en limiter til både minutsektionen og timesektionen, hvor begge blev lavet med \texttt{IF/ELSE} statements. Minutdelen blev begrænset til \texttt{5} for 10'erne og \texttt{9} for 1'erne på minutsektionen.

Timesektionen blev ligeledes begrænset, men med flere argumenter. Først blev 10'er sektionen begrænset til \texttt{2}. Derefter satte vi en betingelse op, der skulle blokere 1'erne i at tælle op fra \texttt{3}, når 10'erne viste \texttt{2}.
Dette endte med en total maksimalværdi på \texttt{23:59}.

% Results
\subsubsection{Resultater}

\texttt{Alarm watch} blev testet i \coderef[testbench]{ex06/vhdl/test_bench.vhd}, hvor også pin mappings ses. 
\includecode[testbench]{ex06/vhdl/test_bench.vhd}{Alarm watch test bench}{linerange={40-61}}

Alarmfunktionaliten blev testet med en alarm sat til \texttt{10:54}, hvoraf resultatet ses i \figref{ex6-4-fotos}.
\begin{figure}[H]
    \centering
    \dsdsubfig{ex6-4-TimeBelow}{Tiden er ikke inden for alarmintervallet.}{0.8}
    \dsdsubfig{ex6-4-TimeSet}{Alarmtiden sættes}{0.8}
    \dsdsubfig{ex6-4-TimeEqual}{Tiden er det samme som alarmtiden.}{0.8}
    \dsdsubfig{ex6-4-TimeOver}{Tiden er ikke inden for alarmintervallet}{0.8}
    \caption{Fotos der viser de forskellige tilstande for alarmen}\label{fig:ex6-4-fotos}
\end{figure}

Funktionaliteten i uret var meget lig den tidligere opgave - da det jo blot er  \coderef[watch1]{ex06/vhdl/watch.vhd} med flere elementer bygget uden på.

% Discussion
\subsubsection{Diskussion}
Input limiteren voldte problemer i den forstand, det ikke lykkedes at få den til at virke helt korrekt. Begrænsningen fungerede uden problemer i minutsektionen, men i timesektionen kunne vi ikke få begrænsningen til at fungere hvis det \emph{kun} var \texttt{bin\_hrs1} der var aktiv. Her kunne resultater som \texttt{[1F:59]} fremkomme - hvilket jo tydeligvis er en fejl --- som ligger i den række \texttt{IF/ELSE} statements der er kaldt. Vi valgte at fjerne den sektion der skabte problemer og beholdte den del der fungerede. Dette må være vores første \emph{bug}. 

% Conclusion
\subsubsection{Konklusion}
   Ved tænke tilbage på funktionaliteten af de moduler vi tidligere har bygget, og bruge dem i nye designs,  har vi haft mulighed for at skabe større og mere komplekse projekter end hidtil ---  Her har vi har her lavet et 24 timers ur (med en enkelt fejl), i formattet \texttt{HH:MM:SS}, der har en alarmfunktion med visuelt feedback - denne opererer i spændet \texttt{[00:00]-[23:59]}.
}