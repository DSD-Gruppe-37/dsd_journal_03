{
\newcommand{\labelprefix}{{src5-4}}    
\subsection{8-input NAND Gate}

% Intro 
\subsubsection{Introduktion}

I denne opgave designede vi en \textit{n}-input \texttt{NAND}-gate. 
Formålet med øvelsen er at udnytte VHDL's loopfunktion, og desuden afprøve \texttt{generic} funktionerne. Sidst nævnte muliggøre \textit{n}-inputtet, da vi med denne selv kan vælge antallet af inputs. 


% Design
\subsubsection{Design og implementering}

For at lave en 8-input \texttt{NAND}, satte vi \texttt{generic} størrelsen til \texttt{+8}.

\includecode[\labelprefix]{ex05/vhdl/nand8.vhd}{Generic input \texttt{NAND} entity og arkitektur}{}

% Results
\subsubsection{Resultater}


\dsdtab{
    \begin{tabular}{p{4cm}p{4cm}p{4cm}}\toprule
        Input & Forventet output & Reelt output \\\midrule
        \texttt{00000000} & \texttt{1} & \texttt{1} \\\midrule
        \texttt{01111110} & \texttt{0} & \texttt{0} \\\midrule
        \texttt{00110011} & \texttt{0} & \texttt{0} \\\bottomrule
    \end{tabular}
    }{Test case for 8 input \texttt{NAND}}{eightInputNand}
    


\dsdfig{ex5-4-8_bit_NAND}{0.8\linewidth}{RTL view: 8 bit \texttt{NAND}}


\dsdfig{ex5-4-32_bit_NAND}{0.5\linewidth}{RTL view: 32 bit \texttt{NAND}}

% Discussion
\subsubsection{Diskussion}
% Conclusion
\subsubsection{Konklusion}}