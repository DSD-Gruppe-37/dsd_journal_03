{
\newcommand{\labelprefix}{src6-2}

\subsection{Clock - One Digit}

% Intro 
\subsubsection{Introduktion}

Denne opgave handler om at skabe en \emph{synchronous clock generator} med \emph{asynchronous reset} som baseret på \texttt{speed} og \texttt{CLOCK\_50} kan tælle i med en konstant hastighed hhv. 1s og 500ms.

% Design
\subsubsection{Design og implementering}

\begin{figure}[H]
    \centering
    \dsdsubfig{ex6-2-clkRTL}{RTL view: Clock generator}{}
    \dsdsubfig{ex6-2-clkPOSTMAP}{Postmap view: Clock generator - for stort til at vise i sin helhed. Viser effektivitekten af VHDL}{0.8}
    \caption{Netlist views af clockgeneratoren}\label{fig:ex6-2-netlist}
\end{figure}
% \dsdfig{ex6-2-clkRTL}{0.8\linewidth}{RTL view: Clock generator}
% \dsdfig{ex6-2-clkPOSTMAP}{0.8\linewidth}{Postmap view: Clock generator - for stort vise korrekt. Viser effektivitekten af VHDL}

Implementationen af \emph{clock generatoren} ses i \coderef[\labelprefix-ent]{ex06/vhdl/clock_gen.vhd} og \coderefsimple[\labelprefix-arch]{ex06/vhdl/clock_gen.vhd}.
For at reagere på et clock signal skriver vi igen arkitekturen som behavioural style. For at sikre at der kun tælles på en rising edge vil vi bruge funktionen \texttt{rising\_edge()} fra \texttt{STD\_LOGIC\_1164} biblioteket.

På en \emph{rising edge} opdateres den maksimale tæller-værdi baseret på nuværende \texttt{speed}. \emph{clock generatoren} nulstilles overskrider den valgte værdi, og ellers tæller den op.

For at udregne maksimalværdien, gjorde vi flg:
\begin{equation}
    \mathtt{speedSelector} = frequency \cdot duration
\end{equation}
\begin{equation}
    50 \cdot 10^6 = \frac{1}{50 \mathrm{MHz}} \cdot 1\textrm{s}
\end{equation}
\begin{equation}
    25 \cdot 10^6 = \frac{1}{50 \mathrm{MHz}} \cdot 0.5\textrm{s}
\end{equation}

\includecode[\labelprefix-ent]{ex06/vhdl/clock_gen.vhd}{Clock generatoren entity og start på arkitekturen (1 af 2)}{linerange={1-30}}

\subsubsection*{Hastighed på clock}
Ved at lave \texttt{speedSelector} som en variable integer, har vi mulighed for at skifte hastighed på \emph{clock}'en. Denne er lavt en \texttt{case}statemetent, hvilket åbner for en mulighed for udvidelser af selektoren --- der kan let laves flere muligheder, ved at tilføje flere \emph{bits} til \texttt{speed}inputtet. 

Under senere test af de forskellige moduler blev \texttt{'0'} valget sat til: \texttt{10e3} hvilket gav en clockperiode på 200µs. 

\includecode[\labelprefix-arch]{ex06/vhdl/clock_gen.vhd}{Clock generatoren speed select og reset (2 af 2)}{firstline=32}

\subsubsection*{Clock logikken}
For hver \emph{rising\_edge} bliver der tjekket om \texttt{ClkDivCounteren} har nået \texttt{speedSelector}værdien.

Hvis den er større end eller lig \texttt{speedSelector}værdien, bliver den nulstillet og clockoutputtet sættes højt.

Efter nulstilling vil \texttt{ClkDivCounteren} givet være lavere end \texttt{speedSelector}værdien, og den inkrementeres med 1, og outputtet holdes lavt.

Ovenstående process gentages ved en frekvens af 50MHz og sker hele tiden, meget hurtigt og er en synkronproces da det styres af en \emph{rising\_edge} drevet proces.

\subsubsection*{Reset logikken}
For at kunne nulstille Counteren UDEN for en\emph{rising\_edge()} implementeres en hard reset.

Denne undersøger om \texttt{reset} knappen er aktiveret. Hvis dette er tilfældet nulstilles counteren og clockoutputtet sættes lavt.

% Results
\subsubsection{Resultater}

\Figref{ex6-2-clkRTL} viser noget af det interessante ved VHDL synthesizering --- Clock generatorenr består réelt set af 6 elementer: 2 logiske elementer, \emph{Add} og \emph{Less than}, 1 enkelt multiplexer, en \emph{OR-gate} med et inveret input, og så to \emph{flipflop}-kredse, hvor af den ene fungere som \emph{clockoutput}. \Figref{ex6-2-clkPOSTMAP} viser effektivitekten af VHDL's synthesizer og konceptet bag \emph{FPGA}boards hardwaremuligheder.

% Discussion
\subsubsection{Diskussion}
Clockgeneratoren virker efter hensigten, og tæller uden problemer op.

% Conclusion
\subsubsection{Konklusion}

Vha. \emph{VHDL}'s biblioteker  \texttt{STD\_LOGIC\_1164} og \texttt{NUMERIC\_STD} kan en clock generator implementeres på et \emph{FPGA} board. Denne kan enkelt sættes til en specfik frekvens.
}