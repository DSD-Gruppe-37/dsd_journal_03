{
    \newcommand{\labelprefix}{{src5-4}}    

    \subsection{8-input NAND Gate}

    % Intro 
    \subsubsection{Introduktion}

    I denne opgave skal der implementeres en 8-input NAND-gate. Dette giver anledning til at afprøve VHDLs \texttt{FOR-LOOP}. Desuden er det også en anledning til at bruge \texttt{generics} for at implementere en \textit{n}-input NAND-gate.

    % Design
    \subsubsection{Design og implementering}

    I vores design er vi sprunget direkte til en generisk implementation. Dette ses på \coderef[\labelprefix]{ex05/vhdl/nand_8.vhd} hvor det på linje 2 ses at vi har brugt \texttt{GENERIC} til at gøre denne entity skalerbar.
    Den generiske værdi har en default på 8, således at NAND-gaten er 8-input medmindre andet specificeres.
    Værdi \texttt{bits} bruges til at sætte input signal størrelsen på linje 4 og til at definere vores loop range på linje 16.

    \includecode[\labelprefix]{ex05/vhdl/nand_8.vhd}{Generic input NAND-gate entity og arkitektur}{firstline=5}

    % Results
    \subsubsection{Resultater}

    NAND-gaten kobles til switches og en LED i en testbench som set på \coderef[\labelprefix]{ex05/vhdl/nand_8_tester.vhd}.

    \includecode[\labelprefix]{ex05/vhdl/nand_8_tester.vhd}{Testbench til 8-input NAND-gate}{firstline=6}

    Desuden har vi lavet en funktionel simulation for at teste at alle kombinationer af inputs giver det rigtige resultat.
    % TODO billede af funktionel simulation

    \dsdtab{
        \begin{tabular}{p{4cm}p{4cm}p{4cm}}\toprule
            Input & Forventet output & Reelt output \\\midrule
            \texttt{00000000} & \texttt{1} & \texttt{1} \\\midrule
            \texttt{01111110} & \texttt{0} & \texttt{0} \\\midrule
            \texttt{00110011} & \texttt{0} & \texttt{0} \\\bottomrule
        \end{tabular}
        }{Test case for 8 input \texttt{NAND}}{eightInputNand}
        


    \dsdfig{ex5-4-8_bit_NAND}{0.8\linewidth}{RTL view: 8 bit \texttt{NAND}}


    \dsdfig{ex5-4-32_bit_NAND}{0.5\linewidth}{RTL view: 32 bit \texttt{NAND}}

    % Discussion
    \subsubsection{Diskussion}

    Vi ser at det vha. \texttt{FOR-LOOP} er nemt at bygge delblokke der gentager det samme logik mange gange, vi ser netop i RTL-viewene at vores loops bliver mappet direkte over på \textit{n}-input AND-gates og en inverter, som det forventes af en NAND-gate.
    Et loop spiller desuden godt sammen med generiske entities, da de generiske værdier kan bruges til at definere ranges.
    Vi ser af den grund at det er nemt med et loop at implementere \textit{n}-input gates af mange forskellige slags!

    % Conclusion
    \subsubsection{Konklusion}

    Vi fik løst opgaven om at implementere en 8-input NAND-gate og desuden gjorde vi dette ved at lave den generisk, så denne NAND-gate kan skaleres til \textit{n}-input, hvor mange man nu lige skal bruge.
}