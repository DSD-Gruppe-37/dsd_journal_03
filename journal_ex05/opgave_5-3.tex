{

\newcommand{\labelprefix}{src5-3}

\subsection{Two Player Guess Game}

% Intro 
\subsubsection{Introduktion}
I denne opgave udvides vores Guess Game således at det kan håndtere to spil simultant, dvs. at begge spillere kan indtaste et hemmeligt tal som den anden spiller skal forsøge at gætte.

% Design
\subsubsection{Design og implementering}
Den overordnede implementering er primært strukturel da vi allerede har lavet en indkapslet \texttt{guess\_game} entity.
En udgave med to spillere kræver blot 2 instanser af vores spil og multiplexing som bestemmer hvilket spil inputs blev sendt til og outputs bliver læst fra.
Desuden tilføjes et ekstra 7-segment display for at vise hvilken spiller der er valgt på nuværende tidspunkt.

\includecode[\labelprefix]{ex05/vhdl/two_player_guess_game.vhd}{Two Player Guess Game Entity}{linerange={5-17}}

\includecode[\labelprefix]{ex05/vhdl/two_player_guess_game.vhd}{[Two Player Guess Game arkitektur og interne signaler]Her ses begyndelsen på \texttt{Two player Guess Game} arkitekturen, samt de signaler der oprettes}{linerange={19-34}}

\includecode[\labelprefix]{ex05/vhdl/two_player_guess_game.vhd}{[Two Player Guess Game arkitektur]Her ses spillerdisplayer, og de to \texttt{Guess Games}}{linerange={36-68}}

\includecode[mux]{ex05/vhdl/two_player_guess_game.vhd}{[Two Player Guess Game arkitektur]Her ses de to multiplexers der holder styr på spillerens valg.}{linerange={69-114}}

De to multiplexers der ses i \coderef[mux]{ex05/vhdl/two_player_guess_game.vhd} blev implementeret vha. \texttt{when} statements, den ene er vist i \coderef[\labelprefix]{ex05/vhdl/mux.vhd}, og den anden er tidliger vist i \coderef[src5-2]{ex05/vhdl/mux.vhd}.

\includecode[\labelprefix]{ex05/vhdl/mux.vhd}{[Two Player Guess Game multiplexers]Her ses de den ene multiplexer der holder styr på spillerens valg.}{linerange={53-70}}


% Results
\subsubsection{Resultater}

\subsubsection*{Latches}
\includecode[\labelprefix]{ex05/latchreport2p.txt}{Udsnit af output report}{}

Som forventet er opstår der latches igen, og outputtet i \coderef[src5-2]{ex05/latchreport1p.txt} fremkommer igen. 

Desuden kommer også \coderef[\labelprefix]{ex05/latchreport2p.txt}


% //TODO Mangler et resultatafsnit...
% Discussion
\subsubsection{Diskussion}
Igen ser vi at det er nemt at opbygge sit design strukturelt.

Ved at bruge \texttt{when} statements, kunne vi implementere den multiplexer der holder styr på hvilken spiller der i brug, direkte i den multiplexer entity der allerede var lavet. Desuden kunne en tidligere multiplexer genbruges.
% Conclusion
\subsubsection{Konklusion}}
% //TODO Mangler en konklusion..
