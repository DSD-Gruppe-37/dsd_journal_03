{

\newcommand{\labelprefix}{src5-3}

\subsection{Two Player Guess Game}

% Intro 
\subsubsection{Introduktion}
Ved at sammensætte to \texttt{Guess Game} entities kan der laves en 2-player udgave. I denne øvelse er der lagt vægt på brugen af \texttt{when} statements til at gøre dette.

% Design
\subsubsection{Design og implementering}

\includecode[\labelprefix]{ex05/vhdl/two_player_guess_game.vhd}{Two Player Guess Game Entity}{linerange={5-17}}

\includecode[\labelprefix]{ex05/vhdl/two_player_guess_game.vhd}{[Two Player Guess Game arkitektur og interne signaler]Her ses begyndelsen på \texttt{Two player Guess Game} arkitekturen, samt de signaler der oprettes}{linerange={19-34}}

\includecode[\labelprefix]{ex05/vhdl/two_player_guess_game.vhd}{[Two Player Guess Game arkitektur]Her ses spillerdisplayer, og de to \texttt{Guess Games}}{linerange={36-68}}

\includecode[mux]{ex05/vhdl/two_player_guess_game.vhd}{[Two Player Guess Game arkitektur]Her ses de to multiplexers der holder styr på spillerens valg.}{linerange={69-114}}

De to multiplexers der ses i \coderef[mux]{ex05/vhdl/two_player_guess_game.vhd} blev implementeret vha. \texttt{when} statements, den ene er vist i \coderef[\labelprefix]{ex05/vhdl/mux.vhd}, og den anden er tidliger vist i \coderef[src5-2]{ex05/vhdl/mux.vhd}.

\includecode[\labelprefix]{ex05/vhdl/mux.vhd}{[Two Player Guess Game multiplexers]Her ses de den ene multiplexer der holder styr på spillerens valg.}{linerange={53-70}}


% Results
\subsubsection{Resultater}
% //TODO Mangler et resultatafsnit...
% Discussion
\subsubsection{Diskussion}
Ved at bruge \texttt{when} statements, kunne vi implementere den multiplexer der holder styr på hvilken spiller der i brug, direkte i den multiplexer entity der allerede var lavet. Desuden kunne en tidligere multiplexer genbruges.
% Conclusion
\subsubsection{Konklusion}}
% //TODO Mangler en konklusion..