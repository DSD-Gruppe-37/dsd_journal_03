{
    \newcommand{\labelprefix}{src6-1}

    \subsection{Counter - One Digit}

    % Intro 
    \subsubsection{Introduktion}

    Denne opgave handler om at skabe en clock tæller som baseret på \texttt{mode} kan tælle enten \texttt{[0:2]}, \texttt{[0:5]} eller \texttt{[0:9]}.

    % Design
    \subsubsection{Design og implementering}

    Implementationen af tælleren ses i \coderef[\labelprefix-ent]{ex06/vhdl/multi_counter.vhd} og \coderefsimple[\labelprefix-arch]{ex06/vhdl/multi_counter.vhd}.
    For at reagere på et clock signal skriver vi arkitekturen som behavioural style, altså en process med signalet \texttt{clk} i sensitivity listen. For at sikre at der kun tælles på en rising edge vil vi bruge funktionen \texttt{rising\_edge()} fra \texttt{STD\_LOGIC\_1164} biblioteket.

    På en \emph{rising edge} opdateres den maksimale tæller-værdi baseret på nuværende \texttt{mode}. Tælleren nulstilles hvis den ville overskride maksimal værdien, ellers tæller den op.

    \includecode[\labelprefix-ent]{ex06/vhdl/multi_counter.vhd}{[Count one digit entity] Interface for multi\_counter entitien}{linerange=5-15}

    \includecode[\labelprefix-arch]{ex06/vhdl/multi_counter.vhd}{[Count one digit arkitektur] Her ses multi\_counter digit arkitekturen, hvor bla. \texttt{mode} og \texttt{reset} implementers}{linerange=17-47, language=vhdl}

    % Results
    \subsubsection{Resultater}

    \includecode[\labelprefix]{ex06/vhdl/multi_counter_tester.vhd}{[Test bench: Multi Counter]Testbench til multi\_counter}{firstline=5}

    % TODO funktionel simulation af counter, vigtigt med skift af MODE fra højt max til lavt max

    % Discussion
    \subsubsection{Diskussion}

    Tælleren virker efter hensigten.
    % TODO og kan den skifte til andre MODES korrekt??

    % Conclusion
    \subsubsection{Konklusion}

    VHDL gør det nemt at lave programmer der reagerer på clock signaler.
    Desuden gør \texttt{IF} statements i en clock'd process det simpelt at implementere synkrone og/eller asynkrone inputs, da der er en indbygget prioritering.
}